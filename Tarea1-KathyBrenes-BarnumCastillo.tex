\documentclass[letterpaper, 10pt, journal]{IEEEtran}
\usepackage{graphicx}
\usepackage{float}
\usepackage{listings}
\usepackage{color}

\usepackage[justification=centering]{caption}
\lstset{frame=tb,
  language=Java,
  aboveskip=3mm,
  belowskip=3mm,
  showstringspaces=false,
  columns=flexible,
  basicstyle={\small\ttfamily},
  numbers=none,
  numberstyle=\tiny\color{gray},
  keywordstyle=\color{blue},
  commentstyle=\color{dkgreen},
  stringstyle=\color{mauve},
  breaklines=true,
  breakatwhitespace=true,
  tabsize=3
}

\begin{document}
\title{Tarea 1 - Reporte de \LaTeX  }
\author{Kathy~Brenes~Guerrero, Barnum~Castillo~Barquero

~\IEEEmembership{
    \begin{center}
        Maestr\'ia en Ciencias de la Computaci\'on, Introducci\'on a la Investigaci\'on, ITCR
    \end{center}
}}% <-this % stops a space

% The paper headers
\markboth{Instituto Tecnol\'ogico de Costa Rica, Introducci\'on a la Investigaci\'on, Agosto~2018}%
{Shell \MakeLowercase{\textit{et al.}}: Tarea 1 - Reporte de \LaTeX  }
\maketitle

\begin{abstract}
One of the biggest issues that an operating system can experience is privilege escalation. Privilege escalation is the act of exploiting a bug, design flaw, or configuration oversight in an operating system or software application to gain elevated access to resources that are normally protected from an application or user. Understanding the weaknesses and flaws of a security level issue for the operating system can help implement better approaches and techniques to improve the software itself. Just because you have updated your computer to the latest update or patch, doesn’t mean that it has been secured. Windows, for example, has a series of vulnerabilities that can affect the operating system and can't be solved by Microsoft because the updates can create incompatibilities with an older system or with some security protocols. The Privilege Escalation technique takes advantage of these vulnerabilities to  gain privileges (access) within a remote  system, in order to run applications and make commands on it. The focus of this paper is to list the vulnerabilities that have been demonstrated by third party systems in different operating system, and provide a technical point of view  on what can be done to avoid these breaches ( vulnerabilities or impacts). An Operation System breach can enable attackers to increase their level of control over target systems, such that they are free to access any data or make any configuration changes. This study reveals the importance of the way in which current systems should be defended from this mechanism.
\end{abstract}
\begin{IEEEkeywords}
Operating System, Penetration Testing, Cybersecurity, Internet of Things.
\end{IEEEkeywords}

\section{Introduction}
To start talking about vulnerabilities, it might be easier to start with past operating systems, specifically MS-DOS and Windows 9x (95, 98 and Me), which were based on MS-DOS.
All software running on an MS-DOS-based system was treated equally. Any program could, literally, do anything. Any program could play directly with the hardware, poke around in memory being used by other programs, or even modify the operating system itself on the fly.

It was not what we\'’d call \'“secure\'” in any way. We suppose the only thing that prevented it from being a security nightmare is that today’s ubiquitous connectivity didn\'’t exist. Compared to what we take for granted today, it was at least cumbersome, and often outright difficult, to get data from one computer to another. Since the kernel can do anything, we refer to it as having more privilege than software running in user-mode. There are a number of different things that can be restricted based on privilege, but memory access is one of the clearer examples

A program running user-mode cannot read and write the memory of another program that happens to be running at the same time. Your web browser, for example, is not able to peek into the document you’re currently editing in a word processor.

It’s important to understand that this concept of “privilege escalation” matters. Hopefully, understanding the concept — even at a high level and perhaps only partially — will give you some idea that it’s important, why it’s important, and how it relates to the security of your computer.

Knowing that it’s important, the single most important thing you can do to avoid issues and vulnerabilities that might be characterized as “privilege escalation” issues is to keep your system as up-to-date as possible. As with the recent CPU issue, operating system vendors are quickly putting out patches to avoid it, and it’ll be important for you to have those patches when they come up.

The best way to do that is to keep whatever OS you run — set to update automatically.Being always aware that keeping the operating system updated helps reduce the risk of these attacks but does not eradicate them 100%.

\section{Historia del \LaTeX  }
TeX es el programa original de composici\'on matem\'atica desarrollado alrededor de 1980 por Donald Knuth  para la composici\'on tipogr\'afica digital de alta calidad.\cite{[3]} TEX es un lenguaje de bajo nivel con el que las computadoras pueden trabajar, pero a la mayor\'ia de las personas les resultar\'ia dif\'icil usarlo; entonces \LaTeX   ha sido desarrollado para hacerlo m\'as f\'acil.\cite{[2]}.Se pronuncia como "tecnolog\'ia" como en alta tecnolog\'ia. La X es la letra griega chi, que hace que el sonido "ch" aparezca al final de "tech". En la documentaci\'on original para TeX, hay mucha discusi\'on acerca de "pegar" objetos juntos y de "estirar" el espacio entre los objetos que componen una p\'agina. \cite{[3]}

El l\'{a}tex es un producto natural pegajoso que forma la base del caucho, y creo que ese es el motivo de la palabra LaTeX. LaTeX est\'{a} construido sobre TeX. Fue escrito a principios de la d\'{e}cada de 1980 por Leslie Lamport. Tiene m\'{a}s funciones de alto nivel integradas que TeX, por lo que tiende a ser m\'{a}s f\'{a}cil de usar. LaTeX no es un intento de sonar en franc\'{e}s; probablemente no sea La TeX, o "The TeX". La raz\'{o}n de la divertida alternancia de letras may\'{u}sculas y min\'{u}sculas es que TeX y LaTeX, cuando se escriben correctamente de esta manera:, son todas letras may\'{u}sculas pero en diferentes tama\~{n}os de puntos y diferentes alturas por encima y por debajo de la l\'{\i}nea de base. Alternando may\'{u}sculas y min\'{u}sculas imita esto. \cite{[3]}
\newline
\LaTeX   se cre\'o para facilitar la producci\'on de libros y art\'iculos de uso general dentro de TeX. Debido a que \LaTeX   es una extensi\'on del sistema de composici\'on tipogr\'afica TeX, tiene la capacidad de TeX para compilar documentos t\'ecnicos que contienen ecuaciones matem\'aticas complejas. Esta caracter\'istica hizo que LaTeX fuera popular entre cient\'ificos e ingenieros. \cite{[1]}
\newline
La producci\'on de un documento \LaTeX   comienza con un archivo de texto que contiene contenido etiquetado con c\'odigos especiales \LaTeX   utilizados para indicar c\'omo se dise\~nar\'a el texto. Cuando el archivo se ejecuta a trav\'es de un procesador \LaTeX  , se producen p\'aginas de composici\'on tipogr\'afica. Debido a que la composici\'on tipogr\'afica \LaTeX   requiere envolver el texto en c\'odigos inform\'aticos complicados, tiene una curva de aprendizaje bastante empinada. Aunque ahora hay programas de software que ayudan a automatizar la creaci\'on de documentos \LaTeX  , un conocimiento pr\'actico de \LaTeX   sigue siendo deseable para este tipo de composici\'on tipogr\'afica.
\newline
\LaTeX   fue uno de los primeros programas de composici\'on tipogr\'afica capaz de producir ecuaciones matem\'aticas complejas. Con los a\~nos se ha utilizado para componer muchas revistas cient\'ificas, matem\'aticas y de ingenier\'ia. La American Mathematical Society (AMS) incluso tiene su propio conjunto de extensiones, llamado AMS-LaTeX, que sus contribuyentes usan para su revista. Pero los programas de autoedici\'on como Quark Inc. de Quark Inc. y FrameMaker de Adobe Systems Incorporated se volvieron m\'as capaces de producir expresiones matem\'aticas complejas, \LaTeX   se hizo menos popular.\cite{[1]}
La versi\'on actual de \LaTeX   es \LaTeX 2e. \cite{[2]}


\section{Usos acad\'emicos, extensi\'on, importancia}
\LaTeX    es un sistema de preparaci\'on de documentos para producir documentos de aspecto profesional, no es un procesador de textos. Es particularmente adecuado para producir documentos largos y estructurados, y es muy bueno para escribir ecuaciones. Est\'a disponible como software libre para la mayor\'ia de los sistemas operativos.
\newline
Si est\'a acostumbrado a producir documentos con Microsoft Word, encontrar\'a que \LaTeX   es un estilo de trabajo muy diferente. Microsoft Word es \''Lo que ves es lo que obtienes\'' (WYSIWYG), esto significa que puedes ver c\'omo se ver\'a el documento final mientras escribes. Cuando trabaje de esta manera, probablemente realice cambios en la apariencia del documento (como espacios entre l\'ineas, encabezados, saltos de p\'agina) mientras escribe. Con \LaTeX   no ver\'a c\'omo se ver\'a el documento final mientras lo est\'a escribiendo; esto le permite concentrarse en el contenido m\'as all\'a de la apariencia.
Para producir esto en la mayor\'ia de los sistemas de tipograf\'ia o procesamiento de textos, el autor deber\'ia decidir qu\'e dise\~no usar, por lo que seleccionar\'ia (digamos) 18pt Times Roman para el t\'itulo, 12pt Times Italic para el nombre, y as\'i sucesivamente. Esto tiene dos resultados: los autores pierden su tiempo con los dise\~nos; y muchos documentos mal dise\~nados. \cite{[2]}
LaTeX se basa en la idea de que es mejor dejar el diseño del documento a los diseñadores de documentos y permitir que los autores continúen con la escritura de documentos. \cite{[2]}
\newline
Un documento \LaTeX   es un archivo de texto sin formato con una extensi\'on de archivo .tex. Se puede escribir en un editor de texto simple como el Bloc de notas, pero la mayor\'ia de las personas encuentran que es m\'as f\'acil usar un editor de \LaTeX   dedicado. Mientras escribe, marque la estructura del documento (t\'itulo, cap\'itulos, subt\'itulos, listas, etc.) con etiquetas. Cuando finaliza el documento, comp\'ilelo; esto significa convertirlo a otro formato. \cite{[2]}
\newline
Existen varios formatos de salida diferentes, pero probablemente el m\'as \'util sea Portable Document Format (PDF), que aparece tal como se imprimir\'a y se puede transferir f\'acilmente entre computadoras. \cite{[2]}
\newline
\textbf{Implementaciones de \LaTeX}
\begin{enumerate}
    \item Composici\'on de art\'iculos de revistas, informes t\'ecnicos, libros y presentaciones de diapositivas.
    \item Control sobre documentos grandes que contienen secciones, referencias cruzadas, tablas y figuras.
    \item Composici\'on tipogr\'afica de f\'ormulas matem\'aticas complejas.
    \item Composici\'on tipogr\'afica avanzada de las matem\'aticas con AMS-LaTeX.
    \item Generaci\'on autom\'atica de bibliograf\'ias e \'indices.
    \item Composici\'on tipogr\'afica multiling\"ue.
    \item Inclusi\'on de obras de arte, y color de proceso o mancha.
    \item Utilizando fuentes PostScript o Metafont.
\end{enumerate}


\subsection{Art\'iculos de revistas}
Text here..
\begin{enumerate}
\item	Some Windows services are configured to run under the Local System user account. A vulnerability such as a buffer overflow (an anomaly where a program, while writing data to a buffer, overruns the buffer\''s boundary and overwrites adjacent memory locations) may be used to execute arbitrary code with privilege elevated to Local System. Alternatively, a system service that is impersonating a lesser user can elevate that user\''s privileges if errors are not handled correctly while the user is being impersonated (e.g. if the user has introduced a malicious error handler)\cite{[6]}.
\item	Under some legacy versions of the Microsoft Windows operating system, the All Users screen saver runs under the Local System account – any account that can replace the current screen saver binary in the file system or Registry can therefore elevate privileges \cite{[6]}.
\item	In certain versions of the Linux kernel it was possible to write a program that would set its current directory to /etc/cron.d, request that a core dump be performed in case it crashes and then have itself killed by another process. The core dump file would have been placed at the program\''s current directory, that is, /etc/cron.d, and cron would have treated it as a text file instructing it to run programs on schedule. Because the contents of the file would be under attacker’s control, the attacker would be able to execute any program with root privileges \cite{[3]}.
\end{enumerate}
Text Here

\section{Estilos de documento}
Text Here
\subsection{Subsection 1}
Example...
\lstset{language=Java}
\begin{lstlisting}
uname -a
cat /proc/version
cat /etc/issue
\end{lstlisting}

\subsection{Subsection 2}
Text here...
\begin{enumerate}
\item Check which processes are running
\lstset{language=Java}
\begin{lstlisting}
# Metasploit
ps
# Linux
ps aux
\end{lstlisting}
\end{enumerate}


\section{C\'omo hacer: p\'arrafos, efectos de letra, tildes, t\'itulos, subt\'itulos, referencias, marcas de agua, headers y footers, manejo de saltos de p\'agina, columnas de la p\'agina, etc.}
\subsection{Subsection 1}
Text here..

\section{Manejo de tablas}
\subsection{Subsection 1}
Text here..

\section{Manejo de figuras y gr\'aficos}
\subsection{Subsection 1}
Text here.

\section{Manejo de figuras al lado de tablas (minipage)}
\subsection{Subsection 1}
Text here.

\section{Ecuaciones matem\'aticas}
\subsection{Subsection 1}
Text here.

\section{Manejo de colores}
\subsection{Subsection 1}
Text here.


\begin{thebibliography}{1}
\bibitem{[1]}The Editors of Encyclopaedia Britannica  (2013) \emph{LaTeX COMPUTER PROGRAMMING LANGUAGE} [Blog post]. Consultado desde https://www.britannica.com/technology/LaTeX-computer-programming-language
\bibitem{[2]} \emph{Introduction to LaTeX.} (2018). Consultado desde https://www.latex-project.org/about/
\bibitem{[3]} \emph{Basic description of file types and how LaTeX works.} (2018). Consultado desde http://personal.bgsu.edu/~zirbel/5920/latex/latex\_basics.htm



\end{thebibliography}
\end{document}






