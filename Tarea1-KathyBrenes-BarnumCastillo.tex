\documentclass[letterpaper, 10pt, journal]{IEEEtran}
\usepackage{graphicx}
\usepackage{float}
% Paquete para las lineas de codigo
\usepackage{listings}
% Paquetes utilizados en la seccion de tablas
\usepackage{color}
\usepackage{multirow}
\usepackage[table,xcdraw]{xcolor}
\usepackage[justification=centering]{caption}
% Paquete para referenciar figuras
\usepackage{nameref}
\graphicspath{ {./Images/} }
% Paquetes para las funciones matematicas
\usepackage{amsmath}
\usepackage{amssymb}

\lstset{frame=tb,
  language=Java,
  aboveskip=3mm,
  belowskip=3mm,
  showstringspaces=false,
  columns=flexible,
  basicstyle={\small\ttfamily},
  numbers=none,
  numberstyle=\tiny\color{gray},
  keywordstyle=\color{blue},
  commentstyle=\color{dkgreen},
  stringstyle=\color{mauve},
  breaklines=true,
  breakatwhitespace=true,
  tabsize=3
}

\begin{document}
\title{Tarea 1 - Reporte de \LaTeX}
\author{Kathy~Brenes~Guerrero, Barnum~Castillo~Barquero

~\IEEEmembership{
    \begin{center}
        Maestr\'ia en Ciencias de la Computaci\'on, Introducci\'on a la Investigaci\'on, ITCR
    \end{center}
}}% <-this % stops a space

% The paper headers
\markboth{Instituto Tecnol\'ogico de Costa Rica, Introducci\'on a la Investigaci\'on, Agosto~2018}%
{Shell \MakeLowercase{\textit{et al.}}: Tarea 1 - Reporte de \LaTeX}
\maketitle

\begin{abstract}
One of the biggest issues that an operating system can experience is privilege escalation. Privilege escalation is the act of exploiting a bug, design flaw, or configuration oversight in an operating system or software application to gain elevated access to resources that are normally protected from an application or user. Understanding the weaknesses and flaws of a security level issue for the operating system can help implement better approaches and techniques to improve the software itself. Just because you have updated your computer to the latest update or patch, doesn’t mean that it has been secured. Windows, for example, has a series of vulnerabilities that can affect the operating system and can't be solved by Microsoft because the updates can create incompatibilities with an older system or with some security protocols. The Privilege Escalation technique takes advantage of these vulnerabilities to  gain privileges (access) within a remote  system, in order to run applications and make commands on it. The focus of this paper is to list the vulnerabilities that have been demonstrated by third party systems in different operating system, and provide a technical point of view  on what can be done to avoid these breaches ( vulnerabilities or impacts). An Operation System breach can enable attackers to increase their level of control over target systems, such that they are free to access any data or make any configuration changes\cite{Williams}. This study reveals the importance of the way in which current systems should be defended from this mechanism.
\end{abstract}
\begin{IEEEkeywords}
Operating System, Penetration Testing, Cybersecurity, Internet of Things.
\end{IEEEkeywords}

\section{Introduction}
To start talking about vulnerabilities, it might be easier to start with past operating systems, specifically MS-DOS and Windows 9x (95, 98 and Me), which were based on MS-DOS.
All software running on an MS-DOS-based system was treated equally. Any program could, literally, do anything. Any program could play directly with the hardware, poke around in memory being used by other programs, or even modify the operating system itself on the fly.

It was not what we\'’d call \'“secure\'” in any way. We suppose the only thing that prevented it from being a security nightmare is that today’s ubiquitous connectivity didn\'’t exist. Compared to what we take for granted today, it was at least cumbersome, and often outright difficult, to get data from one computer to another. Since the kernel can do anything, we refer to it as having more privilege than software running in user-mode. There are a number of different things that can be restricted based on privilege, but memory access is one of the clearer examples

A program running user-mode cannot read and write the memory of another program that happens to be running at the same time. Your web browser, for example, is not able to peek into the document you’re currently editing in a word processor.

It’s important to understand that this concept of “privilege escalation” matters. Hopefully, understanding the concept — even at a high level and perhaps only partially — will give you some idea that it’s important, why it’s important, and how it relates to the security of your computer.

Knowing that it’s important, the single most important thing you can do to avoid issues and vulnerabilities that might be characterized as “privilege escalation” issues is to keep your system as up-to-date as possible. As with the recent CPU issue, operating system vendors are quickly putting out patches to avoid it, and it’ll be important for you to have those patches when they come up.

The best way to do that is to keep whatever OS you run — set to update automatically.Being always aware that keeping the operating system updated helps reduce the risk of these attacks but does not eradicate them 100%.

\section{Historia del \LaTeX}


\section{Usos acad\'emicos, extensi\'on, importancia}
Text here ...

\subsection{SubSection 1...}
Text here..
\begin{enumerate}
\item	Some Windows services are configured to run under the Local System user account. A vulnerability such as a buffer overflow (an anomaly where a program, while writing data to a buffer, overruns the buffer\''s boundary and overwrites adjacent memory locations) may be used to execute arbitrary code with privilege elevated to Local System. Alternatively, a system service that is impersonating a lesser user can elevate that user\''s privileges if errors are not handled correctly while the user is being impersonated (e.g. if the user has introduced a malicious error handler)\cite{[6]}.
\item	Under some legacy versions of the Microsoft Windows operating system, the All Users screen saver runs under the Local System account – any account that can replace the current screen saver binary in the file system or Registry can therefore elevate privileges \cite{[6]}.
\item	In certain versions of the Linux kernel it was possible to write a program that would set its current directory to /etc/cron.d, request that a core dump be performed in case it crashes and then have itself killed by another process. The core dump file would have been placed at the program\''s current directory, that is, /etc/cron.d, and cron would have treated it as a text file instructing it to run programs on schedule. Because the contents of the file would be under attacker’s control, the attacker would be able to execute any program with root privileges \cite{[3]}.
\end{enumerate}
Text Here

\section{Estilos de documento}
Text Here
\subsection{Subsection 1}
Example...
\lstset{language=Java}
\begin{lstlisting}
uname -a
cat /proc/version
cat /etc/issue
\end{lstlisting}

\subsection{Subsection 2}
Text here...
\begin{enumerate}
\item Check which processes are running
\lstset{language=Java}
\begin{lstlisting}
# Metasploit
ps
# Linux
ps aux
\end{lstlisting}
\end{enumerate}


\section{C\'omo hacer: p\'arrafos, efectos de letra, tildes, t\'itulos, subt\'itulos, referencias, marcas de agua, headers y footers, manejo de saltos de p\'agina, columnas de la p\'agina, etc.}
\subsection{Subsection 1}
Text here..

\section{Manejo de Tablas}
Durante esta secci\'{o}n se estar\'{a}n discutiendo el atributo de Latex de creaci\'{o}n de tablas, partiendo del ejemplo de \textquotedblleft{}Tabla b\'{a}sica\textquotedblright{} incluyendo su c\'{o}digo en Latex hasta las diferentes caracter\'{\i}sticas para personalizarla seg\'{u}n las necesidades.

\subsection{Tabla B\'asica}
La siguiente tabla refleja la implementaci\'{o}n m\'{a}s b\'{a}sica, por razones de claridad se le ha agregado las l\'{\i}neas negras.

\begin{table}[H]\centering
\begin{tabular}{|l c|r|}
\hline
Columna 1 & Columna 2 & Columna 3 \\ \hline
Fila 11   & Fila 12   & Fila 13   \\ \hline
Fila 21   & Fila 22   & Fila 23   \\ \hline
\end{tabular}
\end{table}

El c\'{o}digo de la tabla es el siguiente:

\lstset{language=Java}
\begin{lstlisting}
\begin{table}[H]\centering
\begin{tabular}{|l c|r|}
\hline
Columna 1 & Columna 2 & Columna 3 \\ \hline
Fila 11   & Fila 12   & Fila 13   \\ \hline
Fila 21   & Fila 22   & Fila 23   \\ \hline
\end{tabular}
\end{table}
\end{lstlisting}

El comienzo de la tabla se representa con \textbackslash begin\textbackslash\{tabular\}, dentro de esta secci\'{o}n se describe el contenido de la tabla (Columna 1, Columna 2, etc), la separaci\'{o}n de la columna se da por \& y el de cada fila por \textbackslash\textbackslash. 

La alineaci\'on del texto se da por \{\textbar l\textbar c\textbar r\textbar\} siendo l = left, c = center, r = right y los s\'{\i}mbolos pipe (\textquotedblleft{}\textbar{}\textquotedblright{}) representa el delineado de las columnas, para delinear las l\'ineas horizontales se hace uso del \textbackslash{}hline al final de cada fila.

\subsection{Agregar columnas y filas}

Tomando el ejemplo de tabla b\'asica y agreg\'andole una nueva columna y una nueva fila da como resultado el siguiente c\'odigo.

\lstset{language=Java}
\begin{lstlisting}
\begin{table}[H]\centering
\begin{tabular}{|l|l|l|l|}
\hline
Columna 1  & Columna 2 & Columna 3 & C.Nueva \\ \hline
Fila 11    & Fila 12   & Fila 13   & \\ \hline
Fila 21    & Fila 22   & Fila 23   & \\ \hline
Nueva fila &           &           & \\ \hline
\end{tabular}
\end{table}
\end{lstlisting}

Las diferencias notorias como resultado de agregar la nueva fila y columna son \textquotedblleft{}Nueva fila \& \& \& \& \textbackslash{}\textbackslash{}hline\textquotedblright{} al final del c\'{o}digo que representa a la fila y en todas las columnas se ha agregado un \& dando la siguiente tabla.

\begin{table}[H]\centering
\begin{tabular}{|l|l|l|l|}
\hline
Columna 1  & Columna 2 & Columna 3 & C.Nueva \\ \hline
Fila 11    & Fila 12   & Fila 13   & \\ \hline
Fila 21    & Fila 22   & Fila 23   & \\ \hline
Nueva fila &           &           & \\ \hline
\end{tabular}
\end{table}

\subsection{M\'utiples columnas}
Para el proceso de combinar columnas en una misma fila es necesario el comando:

\lstset{language=Java}
\begin{lstlisting}
\multicolumn{ancho}{alineamiento}{contenido}
\end{lstlisting}

Teniendo en cuenta la l\'{\i}nea anterior y si lo agregamos al c\'{o}digo de tabla b\'{a}sica resulta en:

\begin{table}[H]\centering
\begin{tabular}{|l|l|l|}
\hline
Columna 1 & Columna 2  & Columna 3 \\ \hline
\multicolumn{2}{|l|}{Fila 11 + fila 12} & Fila 13 \\ \hline
Fila 21   & Fila 22    & Fila 23   \\ \hline
\end{tabular}
\end{table}

Conociendo la funci\'{o}n del comando multicolumn, podemos ver el c\'{o}digo agregado al de tabla b\'{a}sica y explicar su funcionamiento.

\lstset{language=Java}
\begin{lstlisting}
\begin{table}[H]\centering
\begin{tabular}{|l|l|l|}
\hline
Columna 1 & Columna 2 & Columna 3 \\ \hline
\multicolumn{2}{|l|}{Fila 11 + fila 12} & Fila 13 \\ \hline
Fila 21   & Fila 22   & Fila 23   \\ \hline
\end{tabular}
\end{table}
\end{lstlisting}

\{2\} es la cantidad de columnas que se expandi\'{o} la celda, \{\textbar{}l\textbar{}\} es la delineaci\'{o}n vertical de la celda m\'{a}s la alineaci\'{o}n del texto y el \'{u}ltimo par\'{a}metro es el contenido \textquotedblleft{}Fila 11 + fila12\textquotedblright{}.

\subsection{M\'utiples filas}
La funcionalidad de combinar filas es dada por el paquete.

\lstset{language=Java}
\begin{lstlisting}
\usepackage{multirow}
\end{lstlisting}

La siguiente tabla ejemplifica su uso.

\begin{table}[H]\centering
\begin{tabular}{|l|l|l|}
\hline
Columna 1                & Columna 2 & Columna 3 \\ \hline
\multirow{2}{*}{Fila 11 + fila 21} & Fila 12   & Fila 13   \\ \cline{2-3} 
                         & Fila 22   & Fila 23   \\ \hline
\end{tabular}
\end{table}

\lstset{language=Java}
\begin{lstlisting}
\begin{table}[H]\centering
\begin{tabular}{|l|l|l|}
\hline
Columna 1                & Columna 2 & Columna 3 \\ \hline
\multirow{2}{*}{Fila 11 + fila 21} & Fila 12   & Fila 13   \\ \cline{2-3} 
                         & Fila 22   & Fila 23   \\ \hline
\end{tabular}
\end{table}
\end{lstlisting}

Como se puede apreciar \textbackslash{}multirows tiene los mismos par\'{a}metros que multicolumns con la diferencia en que \{2\} indica la cantidad de filas a ser combinadas.

\subsection{Colores celdas}
Para agregar color a la tabla basta el uso de dos comandos \textbackslash{}rowcolor y \textbackslash{}cellcolor, en el siguiente c\'{o}digo se muestra la ubicaci\'{o}n correspondiente de cada uno.

\lstset{language=Java}
\begin{lstlisting}
\begin{table}[H]\centering
\begin{tabular}{|l|l|l|}
\hline
\rowcolor[HTML]{FFCE93} 
Columna 1                       & Columna 2 & Columna 3 \\ \hline
\cellcolor[HTML]{3166FF}Fila 11 & Fila 12   & Fila 13   \\ \hline
Fila 21                         & Fila 22   & Fila 23   \\ \hline
\end{tabular}
\end{table}
\end{lstlisting}

\begin{table}[H]\centering
\begin{tabular}{|l|l|l|}
\hline
\rowcolor[HTML]{FFCE93} 
Columna 1                       & Columna 2 & Columna 3 \\ \hline
\cellcolor[HTML]{3166FF}Fila 11 & Fila 12   & Fila 13   \\ \hline
Fila 21                         & Fila 22   & Fila 23   \\ \hline
\end{tabular}
\end{table}

Se puede mostrar que ambos comandos hacen uso del c\'{o}digo del color, pero se diferencian en que \textbackslash{}cellcolor s\'{o}lo es aplicado a una celda y \textbackslash{}rowcolor a toda la fila.

\section{Manejo de figuras y gr\'aficos}
En esta secci\'{o}n del documento se explica las funcionalidades de Latex con respecto al manejos de gr\'{a}ficos.

\subsection{Directorio de im\'agenes}
Cuando se agrega una referencia a una imagen en un documento .tex, esta cuando sea compilada por el Latex buscara a la imagen en el directorio en el que se encuentra el .tex. A continuaci\'{o}n, se presentan las formas en las que se puede modificar dicho directorio.

\lstset{language=Java}
\begin{lstlisting}
%Ruta relativa al archivo .tex 
\graphicspath{ {./images/} }

%Ruta absoluta
\graphicspath{ {c:/user/images/} }
\end{lstlisting}

Ambas formas tienen que incorporarse al principio del .tex.

\subsection{Modificar tama\~no y rotaci\'on}
La imagen por insertar en el documento tiene que existir en directorio de im\'{a}genes especificado en el Latex. El c\'{o}digo para insertarlo es el siguiente.

\lstset{language=Java}
\begin{lstlisting}
\includegraphics[scale=1]{latex-logo}
\end{lstlisting}

Se recalca que la imagen latex-logo no tiene extensi\'{o}n de archivo. 
La imagen a continuaci\'{o}n es el resultado del c\'{o}digo anterior y ser\'{a} utilizada como punto de referencia para las dem\'{a}s modificaciones.

\includegraphics[scale=0.2]{latex-logo}

En el c\'{o}digo el tama\~{n}o de la imagen puede ser modificado por [scale=1] con un valor mayor o menor, o puede ser cambiado por [width=x cm, height= x cm].

\lstset{language=Java}
\begin{lstlisting}
\includegraphics[width=3cm, height=4cm]{latex-logo}
\end{lstlisting}

\includegraphics[width=3cm, height=4cm]{latex-logo}

El cambio se observa como:

\lstset{language=Java}
\begin{lstlisting}
\includegraphics[width=\textwidth]{latex-logo}
\end{lstlisting}

Como adici\'{o}n a lo anterior, se puede tomar el ancho del texto como referencia para el ancho de la imagen con \textbackslash{}textwidth.\par{}
La modificaci\'{o}n de la orientaci\'{o}n de la imagen se ve cambiado por el par\'{a}metro angle en t\'{e}rminos de 360 grados.

\lstset{language=Java}
\begin{lstlisting}
\includegraphics[scale=0.5, angle=45]{latex-logo}
\end{lstlisting}
\includegraphics[scale=0.1, angle=45]{latex-logo}

\subsection{Posicionamiento}

\lstset{language=Java}
\begin{lstlisting}
\begin{figure}[posicionamiento]
\includegraphics[width=8cm]{latex-logo}
\end{figure}
\end{lstlisting}

Texto 

\begin{table}[H]
\centering
\begin{tabular}{cl}
\hline
\multicolumn{1}{l}{\textbf{Par\'ametro}} & \textbf{Posicionamiento}                                                                                                                                                                      \\ \hline
h                                      & \begin{tabular}[c]{@{}l@{}}Coloque el flotador aqu\'i, es decir, aproximadamente en el\\ mismo punto en el que aparece el texto fuente (sin embargo,\\ no exactamente en el punto).\end{tabular} \\ \hline
t                                      & Posici\'on en la parte superior de la p\'agina.                                                                                                                                                   \\ \hline
b                                      & Posici\'on en la parte inferior de la p\'agina.                                                                                                                                                   \\ \hline
p                                      & Se pone en una p\'agina especial para flotadores solamente.                                                                                                                                     \\ \hline
!                                      & Anula los par\'ametros internos que LaTeX est\'e utilizando.                                                                                                                                      \\ \hline
H                                      & \begin{tabular}[c]{@{}l@{}}Coloca el flotador exactamente en la ubicaci\'on del c\'odigo \\LaTeX. Requiere el paquete flotante. Esto es algo equiva-\\lente a h!.\end{tabular}                     \\ \hline
\end{tabular}
\end{table}
\subsection{Referenciar figuras}
Texto 
\lstset{language=Java}
\begin{lstlisting}
\begin{figure}[h]
\centering
\includegraphics[width=0.25\textwidth]{grafico}
\label{fig:refejem1}
\end{figure}
 
Como puede ver en la figura \ref{fig:refejem1}, en la P.# \pageref{fig:refejem1}.
\end{lstlisting}

Las referencias una vez compiladas se ven como:
\begin{figure}[H]
    \centering
    \includegraphics[width=0.25\textwidth]{grafico}
    \caption{refejem1}
    \label{fig:refejem1}
\end{figure}
 
En la figura \ref{fig:refejem1} en la P.\# \pageref{fig:refejem1}.

\section{Manejo de figuras al lado de tablas (minipage)}
El minipage es una funcionalidad de Latex utilizada para poner lado a lado estructuras que de otra forma ser\'{\i}a dif\'{\i}cil, su armaz\'{o}n en c\'{o}digo se ve como:

\begin{lstlisting}
\begin{minipage}[ajuste]{ancho del minipage}
  Texto ... \ \
  Imagenes ... \ \
  Tablas ... \ \
\end{minipage} 
\end{lstlisting}

Partiendo de esta implementaci\'{o}n se pueden crear diferentes configuraciones.

Ejemplo de tabla + tabla.

\begin{lstlisting}
\begin{minipage}{0.2\textwidth}
\begin{tabular}{|c|c|c|}
\hline
 A & B & C \\
\hline
 1 & 2 & 3  \\
\hline 
 4 & 5 & 6 \\
\hline
\end{tabular}
\end{minipage}
\begin{minipage}{0.2\textwidth}
\begin{tabular}{c|c|c}
 A & B & C \\
\hline
 1 & 2 & 3  \\
\hline 
 4 & 5 & 6 \\
\end{tabular}
\end{minipage}
\end{lstlisting}

\begin{minipage}{0.2\textwidth}
\begin{tabular}{|c|c|c|}
\hline
 A & B & C \\
\hline
 1 & 2 & 3  \\
\hline 
 4 & 5 & 6 \\
\hline
\end{tabular}
\end{minipage}
\begin{minipage}{0.2\textwidth}
\begin{tabular}{c|c|c}
 A & B & C \\
\hline
 1 & 2 & 3  \\
\hline 
 4 & 5 & 6 \\
\end{tabular}
\end{minipage}
\\
\\
Ejemplo de imagen + texto.

\begin{lstlisting}
\begin{minipage}[t]{0.2\textwidth}
\includegraphics[scale=0.4]{tec-logo}
\end{minipage}
\begin{minipage}{0.2\textwidth}
Tecnologico de CR\\
Tecnologico de CR\\
Tecnologico de CR\\
Tecnologico de CR\\
Tecnologico de CR\\
\end{minipage}
\end{lstlisting}

\begin{minipage}[t]{0.2\textwidth}
\includegraphics[scale=0.4]{tec-logo}
\end{minipage}
\begin{minipage}{0.2\textwidth}
Tecnol\'{o}gico de CR\\
Tecnol\'{o}gico de CR\\
Tecnol\'{o}gico de CR\\
Tecnol\'{o}gico de CR\\
Tecnol\'{o}gico de CR\\
\end{minipage}

Ejemplos de imagen + imagen + imagen.

\begin{lstlisting}
\begin{minipage}[b]{0.15\textwidth}
\includegraphics[width=\textwidth]{tec-logo}
\end{minipage}
\begin{minipage}[t]{0.15\textwidth}
\includegraphics[width=\textwidth]{tec-logo}
\end{minipage}
\begin{minipage}[t]{0.15\textwidth}
\includegraphics[width=\textwidth]{tec-logo}
\end{minipage}
\end{lstlisting}

\begin{minipage}[b]{0.15\textwidth}
\includegraphics[width=\textwidth]{tec-logo}
\end{minipage}
\begin{minipage}[t]{0.15\textwidth}
\includegraphics[width=\textwidth]{tec-logo}
\end{minipage}
\begin{minipage}[t]{0.15\textwidth}
\includegraphics[width=\textwidth]{tec-logo}
\end{minipage}

\section{Ecuaciones matem\'aticas}
En este segmento se presentan las caracter\'{\i}sticas b\'{a}sicas para generar f\'{o}rmulas matem\'{a}ticas en Latex.

\subsection{Modo matem\'{a}ticas}
Latex permite dos tipos de f\'{o}rmulas: inline y display, la primera permite escribir la f\'{o}rmula en l\'{\i}nea con el texto y la segunda lo opuesto, por lo que cuando se escriba ocupar\'{a} una l\'{\i}nea completa por si sola.\par
Las f\'{o}rmulas \textquotedblleft{}inline\textquotedblright{} se caracterizan por estar delimitadas por \textdollar{}formula\textdollar{} y las \textquotedblleft{}display\textquotedblright{} por \textdollar{}\textdollar{}formula\textdollar{}\textdollar{} o \textbackslash{}begin\{equation\} formula \textbackslash{}end\{equation\}. Cabe recalcar que en el modo inline las expresiones son compresas para seguir con la l\'{\i}nea del texto.

Ejemplo de inline:

\begin{lstlisting}
La formula $E=MC^2$ fue formulada por Einstein anos despues de presentar su teoria de la relativadad.
\end{lstlisting}

La f\'{o}rmula $E=MC^2$ fue formulada por Einstein a\~{n}os despu\'{e}s de presentar su teor\'{i}a de la relativadad.

Ejemplo de display:
\begin{lstlisting}
La formula $$E=MC^2$$ fue formulada por Einstein anos despues de presentar su teoria de la relativadad.En unidades naturales ($c$ = 1), representa la identidad

\begin{equation}
E = m
\end{equation}
\end{lstlisting}
La f\'{o}rmula $$E=MC^2$$ fue formulada por Einstein a\~{n}os despu\'{e}s de presentar su teor\'{i}a de la relativadad.En unidades naturales ($c$ = 1), representa la identidad

\begin{equation}
E = m
\end{equation}

\subsection{S\'{i}mbolos especiales}
La utilizaci\'{o}n de s\'{\i}mbolos especiales se da por el modo matem\'{a}tica explicado en el punto anterior, solamente basta escribir \textbackslash{}simbolo dentro de los \textdollar{} \textdollar{}.

\begin{lstlisting}
$$\delta \alpha$$
\end{lstlisting}

$$\delta \alpha$$

\subsection{Fracciones}

Las fracciones pueden ser utilizadas en conjunto con el texto $\frac{1}{2} $ o por si solas con los modos inline y display.

$$\frac{1}{2} $$

Las fracciones son muy vers\'{a}tiles, pudiendo ser anidadas para expresiones complejas.

\begin{lstlisting}
Ejemplo sencillo:

$$\frac{1}{2}$$

Ejemplo anidado:
 
$$ \frac{1+\frac{a}{b}}{1+\frac{1}{1+\frac{1}{a}}} $$
 
\end{lstlisting}

Ejemplo sencillo:

$$\frac{1}{2}$$

Ejemplo anidado:
 
$$ \frac{1+\frac{a}{b}}{1+\frac{1}{1+\frac{1}{a}}} $$
 
\subsection{Operadores}
El manejo de operadores es muy variado por lo que a continuaci\'{o}n se explican los m\'{a}s utilizados.

$$\int_{a}^{b} x^2 dx$$

Las integrales como la anterior son definidas en Latex como:

\begin{lstlisting}
\int_{superior}^{inferior}

$$\int_{a}^{b} x^2 dx$$
\end{lstlisting}

Integrales anidadas pueden resultar más complejas, pero se pueden obtener modificando el inicio de la expresión (int) como en los siguientes ejemplos:
\begin{lstlisting}
$$\iint_V \mu(u,v) \,du\,dv$$
$$\iiint_V \mu(u,v,w) \,du\,dv\,dw$$
$$\idotsint_V \mu(u_1,\dots,u_k) \,du_1 \dots du_k$$
\end{lstlisting}

$$\iint_V \mu(u,v) \,du\,dv$$
$$\iiint_V \mu(u,v,w) \,du\,dv\,dw$$
$$\idotsint_V \mu(u_1,\dots,u_k) \,du_1 \dots du_k$$

Al igual que las integrales, las sumatorias est\'{a}n dadas por:
\begin{lstlisting}
\sum_{superior}^{inferior}

Ejemplo:
Sum $\sum_{n=1}^{\infty} 2^{-n} = 1$ inside text
\end{lstlisting}

$$\sum_{n=1}^{\infty} 2^{-n} = 1$$

En el caso de los l\'{i}mites se utiliza el comando

\begin{lstlisting}
\lim_{lower}

Ejemplo:
$$\lim_{x\to\infty} f(x)$$
\end{lstlisting}

$$\lim_{x\to\infty} f(x)$$

\section{Manejo de colores}
El manejo de colores en Latex est\'{a} dado por los paquetes

\begin{lstlisting}
\usepackage{color}
\usepackage{xcolor}
\end{lstlisting}

Ambos paquetes permiten la manipulaci\'{o}n de colores, pero a la vez tiene la flexibilidad de modificar los colores y secciones a gusto. El siguiente ejemplo muestra los colores predefinidos.

\begin{lstlisting}
\begin{itemize}
\color{blue}
\item Primer item
\item Segundo item
\end{itemize}
 
\noindent
{\color{red} \rule{\linewidth}{0.5mm} }
\end{lstlisting}

\begin{itemize}
\color{blue}
\item Primer item
\item Segundo item
\end{itemize}
 
\noindent
{\color{red} \rule{\linewidth}{0.5mm} }

Los colores red y blue est\'{a} predefinidos en el paquete color, algunos otros ejemplos b\'{a}sicos son.

\begin{lstlisting}
El color del texto puede ser cambiado a \textcolor{red}{rojo}. Tambien puede cambiar el background del \colorbox{BurntOrange}{texto}.
\end{lstlisting}

El color del texto puede ser cambiado a \textcolor{red}{rojo}. Tambien puede cambiar el background del \colorbox{orange}{texto}.\par{}

Para una mayor variedad de colores se pueden crear variables de color d\'{a}ndole una etiqueta a un valor RGB.

\begin{lstlisting}
\definecolor{rosado}{rgb}{0.858, 0.188, 0.478}
\definecolor{gris}{gray}{0.6}
\end{lstlisting}

\definecolor{rosado}{rgb}{0.858, 0.188, 0.478}
\definecolor{gris}{gray}{0.6}

\begin{enumerate}
\item \textcolor{rosado}{Rosado}
\item \textcolor{gris}{Gris}
\end{enumerate}

El definecolor utiliza una de las siguientes 4 opciones para el primer par\'{a}metro.

\begin{enumerate}
\item \textbf{rgb}: rojo, verde, azul. Tres valores separados por comas entre 0 y 1 definen los componentes del color.
\item \textbf{RGB}: lo mismo que rgb, pero los n\'{u}meros son enteros entre 0 y 255.
\item \textbf{cmyk}: cian, magenta, amarillo y blacK. Lista de cuatro n\'{u}meros separados por comas entre 0 y 1 que determina el color de acuerdo con el modelo aditivo utilizado en la mayor\'{\i}a de las impresoras.
\item \textbf{gray}: escala de grises. Un solo n\'{u}mero entre 0 y 1.
\end{enumerate}

Adem\'{a} de modificar el color del texto de puede cambiar el color de la p\'{a}gina con

\begin{lstlisting}
\pagecolor{black}
\end{lstlisting}

\includegraphics[scale=1]{ejemplofondo}

\begin{thebibliography}{1}
\bibitem{a} \emph{Latex-Tutorial.} (2018). Consultado desde https://www.latex-tutorial.com
\bibitem{b} \emph{ShareLatex.} (2018). Consultado desde https://www.sharelatex.com/learn
\bibitem{c} \emph{Sascha-Frank.} (2018). Consultado desde http://www.sascha-frank.com


\end{thebibliography}
\end{document}






